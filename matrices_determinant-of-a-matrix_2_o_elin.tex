\documentclass[12pt]{article}

\usepackage[T2A]{fontenc}      
\usepackage[utf8]{inputenc}   
\usepackage[russian]{babel}   
\usepackage{amsmath}          
\usepackage{amssymb}         
\usepackage{amsthm}           
\usepackage{geometry}         
\usepackage{mathtools}        

\geometry{
    a4paper,
    left=25mm,
    right=20mm,
    top=25mm,
    bottom=25mm
}

\setlength{\parindent}{1.25cm}

\begin{document}

\section*{Задача}

Найти значения параметра $\lambda$, при которых выполняется равенство:
\[
\begin{vmatrix}
4-\lambda & 1 & 0 \\
2 & 6-\lambda & 1 \\
0 & 1 & 4-\lambda
\end{vmatrix}
=0.
\]

Вычислим определитель, разложив его по первой строке:
\[
(4-\lambda)
\begin{vmatrix}
6-\lambda & 1 \\
1 & 4-\lambda
\end{vmatrix}
-
1\cdot
\begin{vmatrix}
2 & 1 \\
0 & 4-\lambda
\end{vmatrix}.
\]

Вычислим миноры:
\[
\begin{vmatrix}
6-\lambda & 1 \\
1 & 4-\lambda
\end{vmatrix}
=(6-\lambda)(4-\lambda)-1
=\lambda^2-10\lambda+23,
\]

\[
\begin{vmatrix}
2 & 1 \\
0 & 4-\lambda
\end{vmatrix}
=2(4-\lambda).
\]

Подставляя, получаем:
\[
(4-\lambda)(\lambda^2-10\lambda+23)-2(4-\lambda)
=(4-\lambda)(\lambda^2-10\lambda+21).
\]

Решаем уравнение:
\[
(4-\lambda)(\lambda^2-10\lambda+21)=0.
\]

Отсюда:
\[
\lambda=4 \quad \text{или} \quad (\lambda-3)(\lambda-7)=0.
\]

\[
\boxed{\lambda=3,\;4,\;7}.
\]

\end{document}
