\documentclass{article}
\usepackage{amsmath}
\usepackage{amssymb}
\usepackage[T2A]{fontenc}
\usepackage[utf8]{inputenc}
\usepackage[russian]{babel}

\begin{document}

\section*{Задача по линейной алгебре}
Даны матрицы:

A = \begin{pmatrix}
2 & 1 \\
-4 & -3
\end{pmatrix}, \quad
B = \begin{pmatrix}
2 & 2 \\
6 & 4
\end{pmatrix}

Решить матричное уравнение:

X A = B

Умножим справа обе части на A^{-1}:

X A \cdot A^{-1} = B \cdot A^{-1}

X = B A^{-1}

\subsection*{1. Находим определитель матрицы A}

\det A = 2 \cdot (-3) - (-4) \cdot 1 = -6 + 4 = -2

\subsection*{2. Находим матрицу алгебраических дополнений A^*}

A^* = \begin{pmatrix}
-3 & 4 \\
-1 & 2
\end{pmatrix}

\subsection*{3. Транспонируем матрицу алгебраических дополнений}

(A^*)^T = \begin{pmatrix}
-3 & -1 \\
4 & 2
\end{pmatrix}

\subsection*{4. Находим обратную матрицу A^{-1}}

A^{-1} = \frac{1}{\det A} \cdot (A^*)^T = -\frac{1}{2} \cdot \begin{pmatrix}
-3 & -1 \\
4 & 2
\end{pmatrix} = \begin{pmatrix}
\frac{3}{2} & \frac{1}{2} \\[4pt]
-2 & -1
\end{pmatrix}

\subsection*{5. Находим матрицу X}

X = B \cdot A^{-1} = \begin{pmatrix}
2 & 2 \\
6 & 4
\end{pmatrix} \cdot \begin{pmatrix}
\frac{3}{2} & \frac{1}{2} \\[4pt]
-2 & -1
\end{pmatrix}

Вычисляем поэлементно:

Первая строка:

2 \cdot \frac{3}{2} + 2 \cdot (-2) = 3 - 4 = -1, \quad
2 \cdot \frac{1}{2} + 2 \cdot (-1) = 1 - 2 = -1

Вторая строка:

6 \cdot \frac{3}{2} + 4 \cdot (-2) = 9 - 8 = 1, \quad
6 \cdot \frac{1}{2} + 4 \cdot (-1) = 3 - 4 = -1

Итого:

X = \begin{pmatrix}
-1 & -1 \\
1 & -1
\end{pmatrix}

\subsection*{Ответ:}

\boxed{X = \begin{pmatrix}
-1 & -1 \\
1 & -1
\end{pmatrix}}

\end{document}
