\documentclass[12pt]{article}
\usepackage[T2A]{fontenc}
\usepackage[utf8]{inputenc}
\usepackage[russian]{babel}
\usepackage{amsmath}
\usepackage{amssymb}
\usepackage{geometry}
\geometry{a4paper, margin=1.5cm}

\setlength{\parindent}{0pt}
\setlength{\parskip}{1em}

\begin{document}

% ========== НАЧАЛО ЗАДАЧИ ==========
\section*{Задача}
Исследовать систему линейных уравнений и найти её решение:

\[
\begin{cases} 
4x + 3y = 4 \\
2x + 2y - 2z = 0 \\
5x + 3y + z = -2 
\end{cases}
\]

\subsection*{Решение}
Расширенная матрица системы \((A|B)\):
\[
(A|B) =
\left( \begin{array}{ccc|c}
4 & 3 & 0 & 4 \\
2 & 2 & -2 & 0 \\
5 & 3 & 1 & -2
\end{array} \right)
\]

\textbf{Приведём матрицу к ступенчатому виду элементарными преобразованиями строк.}

1. Поменяем местами I и II строки для более удобных вычислений
Получаем:
\[
\left( \begin{array}{ccc|c}
2 & 2 & -2 & 0 \\
4 & 3 & -0 & 4 \\
5 & 3 & 1 & -2
\end{array} \right)
\]

2. 
\[
\text{II} \rightarrow \text{II} - 2 \cdot \text{II}
\]
\[
\text{III} \rightarrow 2 \cdot \text{II} - 5 \cdot \text{III}
\]
Получаем:
\[
\left( \begin{array}{ccc|c}
2 & 2 & -2 & 0 \\
0 & -1 & 4 & 4 \\
0 & -2 & 6 & -2
\end{array} \right)
\]

3.
\[
\text{III} \rightarrow \text{III} + 2 \cdot \text{II}
\]
Получаем ступенчатую матрицу:
\[
\left( \begin{array}{ccc|c}
2 & 2 & -2 & 0 \\
0 & -1 & 4 & 4 \\
0 & 0 & -2 & -10
\end{array} \right)
\]

\textbf{Исследование системы:}
\begin{itemize}
    \item Ранг матрицы системы: \( r(A) = 3 \) (количество ненулевых строк)
    \item Ранг расширенной матрицы: \( r(A|B) = 3 \)
    \item Так как \( r(A) = r(A|B) = 3 \), система совместна (имеет единственное решение)
    \item Число неизвестных \( n = 3 \)
    \item Поскольку \( r(A) = n \), система определена (имеет единственное решение)
\end{itemize}

\textbf{Решение системы:}
\[
\begin{cases}
2x + 2y - 2z = 0 \\
-y + 4z = 4 \\
- 2z = -10
\end{cases}
\]

Из третьего уравнения: \(-2z = -10 \ \Rightarrow\ z = 5\)

Из второго уравнения: \(-y + 4 \cdot 5 = 4 \ \Rightarrow\ y = 16\)

Из первого уравнения: \(2x + 2 \cdot 16 - 2 \cdot 5 = 0 \ \Rightarrow\  x = -11\)

\subsection*{Проверка}
Подставим найденные значения в исходные уравнения:
\[
\begin{cases}
4(-11) + 3 \cdot 16 = 4 \\
2(-11) + 2 \cdot 16 - 2 \cdot 5 = 0 \\
5(-11) + 3 \cdot 16 + 5 = -2
\end{cases}
\]
Все равенства выполняются.

\subsection*{Ответ}
Система совместна и определена. Единственное решение:
\[
{x = -11,\quad y = 16,\quad z = 5}
\]

\end{document}

\end{document}
