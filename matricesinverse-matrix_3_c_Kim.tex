\documentclass[12pt]{article}
\usepackage[T2A]{fontenc}
\usepackage[utf8]{inputenc}
\usepackage[russian]{babel}
\usepackage{amsmath}
\usepackage{amssymb}
\usepackage{geometry}
\geometry{a4paper, margin=1.5cm}

\setlength{\parindent}{0pt}
\setlength{\parskip}{1em}

\title{Отработка Коллоквиум 1}
\author{Dianna Kim}
\date{December 2025}

\begin{document}

\maketitle


% ========== НАЧАЛО ЗАДАЧИ ==========
\section*{Задача №3 с)}
Решить систему уравнений методом обратной матрицы:

\[
\begin{cases}
4x + 3y = 4, \\
2x + 2y - 2z = 0, \\
5x + 3y + z = -2.
\end{cases}
\]

\subsection*{Решение}
Запишем систему в матричной форме \( A \mathbf{x} = \mathbf{b} \), где

\[
A = 
\begin{pmatrix}
4 & 3 & 0 \\
2 & 2 & -2 \\
5 & 3 & 1
\end{pmatrix},
\quad
\mathbf{x} = 
\begin{pmatrix}
x \\ y \\ z
\end{pmatrix},
\quad
\mathbf{b} = 
\begin{pmatrix}
4 \\ 0 \\ -2
\end{pmatrix}.
\]

Решение находится по формуле \( \mathbf{x} = A^{-1} \mathbf{b} \), при условии, что \( \det A \ne 0 \).

1. Найдём определитель матрицы \( A \):

\[
\det A =
\begin{vmatrix}
4 & 3 & 0 \\
2 & 2 & -2 \\
5 & 3 & 1
\end{vmatrix}
= 4 \cdot 
\begin{vmatrix}
2 & -2 \\
3 & 1
\end{vmatrix}
- 3 \cdot 
\begin{vmatrix}
2 & -2 \\
5 & 1
\end{vmatrix}
= 4(2 \cdot 1 - (-2) \cdot 3) - 3(2 \cdot 1 - (-2) \cdot 5)
\]
\[
= 4(2 + 6) - 3(2 + 10) = 4 \cdot 8 - 3 \cdot 12 = 32 - 36 = -4 \ne 0.
\]

Обратная матрица существует.

2. Найдём алгебраические дополнения элементов матрицы \( A \):

\[
\begin{aligned}
A_{11} &= +\begin{vmatrix} 2 & -2 \\ 3 & 1 \end{vmatrix} = 8, &
A_{12} &= -\begin{vmatrix} 2 & -2 \\ 5 & 1 \end{vmatrix} = -12, &
A_{13} &= +\begin{vmatrix} 2 & 2 \\ 5 & 3 \end{vmatrix} = -4, \\
A_{21} &= -\begin{vmatrix} 3 & 0 \\ 3 & 1 \end{vmatrix} = -3, &
A_{22} &= +\begin{vmatrix} 4 & 0 \\ 5 & 1 \end{vmatrix} = 4, &
A_{23} &= -\begin{vmatrix} 4 & 3 \\ 5 & 3 \end{vmatrix} = 3, \\
A_{31} &= +\begin{vmatrix} 3 & 0 \\ 2 & -2 \end{vmatrix} = -6, &
A_{32} &= -\begin{vmatrix} 4 & 0 \\ 2 & -2 \end{vmatrix} = 8, &
A_{33} &= +\begin{vmatrix} 4 & 3 \\ 2 & 2 \end{vmatrix} = 2.
\end{aligned}
\]

Составим матрицу алгебраических дополнений и транспонируем её:

\[
\widetilde{A} = 
\begin{pmatrix}
8 & -3 & -6 \\
-12 & 4 & 8 \\
-4 & 3 & 2
\end{pmatrix}.
\]

Тогда обратная матрица:

\[
A^{-1} = \frac{1}{\det A} \, \widetilde{A} = -\frac{1}{4}
\begin{pmatrix}
8 & -3 & -6 \\
-12 & 4 & 8 \\
-4 & 3 & 2
\end{pmatrix}.
\]

3. Найдём решение:

\[
\mathbf{x} = A^{-1} \mathbf{b} = -\frac{1}{4}
\begin{pmatrix}
8 & -3 & -6 \\
-12 & 4 & 8 \\
-4 & 3 & 2
\end{pmatrix}
\begin{pmatrix}
4 \\ 0 \\ -2
\end{pmatrix}
= -\frac{1}{4}
\begin{pmatrix}
8 \cdot 4 + (-3) \cdot 0 + (-6) \cdot (-2) \\
-12 \cdot 4 + 4 \cdot 0 + 8 \cdot (-2) \\
-4 \cdot 4 + 3 \cdot 0 + 2 \cdot (-2)
\end{pmatrix}
\]
\[
= -\frac{1}{4}
\begin{pmatrix}
32 + 0 + 12 \\
-48 + 0 - 16 \\
-16 + 0 - 4
\end{pmatrix}
= -\frac{1}{4}
\begin{pmatrix}
44 \\ -64 \\ -20
\end{pmatrix}
=
\begin{pmatrix}
-11 \\ 16 \\ 5
\end{pmatrix}.
\]

\subsection*{Проверка}
Подставим \( x = -11 \), \( y = 16 \), \( z = 5 \) в исходную систему:

\[
\begin{aligned}
4x + 3y &= 4(-11) + 3(16) = -44 + 48 = 4 \quad \checkmark \\
2x + 2y - 2z &= 2(-11) + 2(16) - 2(5) = -22 + 32 - 10 = 0 \quad \checkmark \\
5x + 3y + z &= 5(-11) + 3(16) + 5 = -55 + 48 + 5 = -2 \quad \checkmark
\end{aligned}
\]

Все уравнения выполнены.

\subsection*{Ответ:}
\[
x = -11, \quad y = 16, \quad z = 5.
\]

\end{document}
