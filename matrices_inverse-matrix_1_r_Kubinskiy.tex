\documentclass[12pt]{article}
\usepackage[utf8]{inputenc}
\usepackage[T2A]{fontenc}
\usepackage[russian]{babel}
\usepackage{amsmath}
\usepackage{amssymb}
\usepackage{geometry}
\geometry{a4paper, margin=2cm}

\begin{document}

\noindent
\textbf{Нахождение обратной матрицы:}


\[
A = 
\begin{pmatrix}
10 & 20 & 30 \\
7 & 18 & 20 \\
-2 & -7 & -5
\end{pmatrix}
\]

\noindent
1)\textbf{Обратная матрица находится по формуле:}

\[
A^{-1} = \frac{1}{|A|} \cdot \text{adj}(A)
\]

\noindent
2)\textbf{Найдем определитель матрицы:}

\begin{align*}
|A| &= 10 \cdot 18 \cdot (-5) + 20 \cdot 20 \cdot (-2) + 7 \cdot (-7) \cdot 30 - 30 \cdot 18 \cdot (-2) \\
    &- 7 \cdot 20 \cdot (-5) - 10 \cdot (-7) \cdot 20 = -900 -800 - 1470 \\
    &+ 1080 + 700 + 1400 = 10
\end{align*}

\noindent
3) $\text{adj}(A)$ — это транспонированная матрица алгебраических дополнений:

\noindent
матрица алгебраических дополнений:

\[
\begin{pmatrix}
50 & -5 & -13 \\
-110 & 10 & 30 \\
-140 & 10 & 40
\end{pmatrix} = B
\]

\noindent
4) транспонируем её:

\[
B^T = 
\begin{pmatrix}
50 & -110 & -140 \\
-5 & 10 & 10 \\
-13 & 30 & 40
\end{pmatrix}
\]

\vspace{1cm}

\noindent
5) находим обратную матрицу:

\[
A^{-1} = \frac{1}{10}\begin{pmatrix}
50 & -110 & -140 \\
-5 & 10 & 10 \\
-13 & 30 & 40
\end{pmatrix} = \begin{pmatrix}
5 & -11 & -14 \\
-0.5 & 1 & 1 \\
-1.3 & 3 & 4
\end{pmatrix}
\]

\noindent
6) проверяем её, умножая на исходную матрицу:

\[
A \cdot A^{-1} = \begin{pmatrix}
10 & 20 & 30 \\
7 & 18 & 20 \\
-2 & -7 & -5
\end{pmatrix} \cdot \begin{pmatrix}
5 & -11 & -14 \\
-0.5 & 1 & 1 \\
-1.3 & 3 & 4
\end{pmatrix} = \begin{pmatrix}
1 & 0 & 0 \\
0 & 1 & 0 \\
0 & 0 & 1
\end{pmatrix}
\]

\noindent
Получилась единичная матрица, значит обратная матрица вычислена верно.

\end{document}
