\documentclass[12pt]{article}
\usepackage[T2A]{fontenc}
\usepackage[utf8]{inputenc}
\usepackage[russian]{babel}
\usepackage{amsmath}
\usepackage{amssymb}
\usepackage{geometry}
\hbadness=10000
\geometry{a4paper, margin=1.5cm}

\setlength{\parindent}{0pt}
\setlength{\parskip}{1em}

\begin{document}

% ========== НАЧАЛО ЗАДАЧИ ==========
\section*{Задача}
Найти обратную матрицу \( A^{-1} \) к матрице \( A \):

\[
A = \begin{pmatrix}
3 & 1 & 10 \\
2 & 5 & 4 \\
-1 & -3 & 1
\end{pmatrix}
\]


\subsection*{Решение}
Для квадратной матрицы \(3\times 3\) обратную матрицу найдём по формуле:
\[
A^{-1} = \frac{1}{\det A} \cdot \operatorname{adj}(A),
\]
где \(\operatorname{adj}(A)\) — присоединённая матрица (транспонированная матрица алгебраических дополнений).

\textbf{Шаг 1. Вычисление определителя матрицы \(A\).}

Разложим определитель по первой строке:
\[
\det A = 3 \cdot \begin{vmatrix} 5 & 4 \\ -3 & 1 \end{vmatrix}
- 1 \cdot \begin{vmatrix} 2 & 4 \\ -1 & 1 \end{vmatrix}
+ 10 \cdot \begin{vmatrix} 2 & 5 \\ -1 & -3 \end{vmatrix}.
\]

Вычислим миноры:
\begin{align*}
\begin{vmatrix} 5 & 4 \\ -3 & 1 \end{vmatrix} &= 5 \cdot 1 - 4 \cdot (-3) = 5 + 12 = 17,\\[4pt]
\begin{vmatrix} 2 & 4 \\ -1 & 1 \end{vmatrix} &= 2 \cdot 1 - 4 \cdot (-1) = 2 + 4 = 6,\\[4pt]
\begin{vmatrix} 2 & 5 \\ -1 & -3 \end{vmatrix} &= 2 \cdot (-3) - 5 \cdot (-1) = -6 + 5 = -1.
\end{align*}

Подставляем:
\[
\det A = 3 \cdot 17 - 1 \cdot 6 + 10 \cdot (-1) = 51 - 6 - 10 = 35.
\]

Так как \(\det A = 35 \neq 0\), обратная матрица существует.

\textbf{Шаг 2. Нахождение матрицы алгебраических дополнений \(C_{ij}\).}

Алгебраическое дополнение: \(C_{ij} = (-1)^{i+j} M_{ij}\), где \(M_{ij}\) — минор элемента \(a_{ij}\).

\begin{align*}
C_{11} &= + \begin{vmatrix} 5 & 4 \\ -3 & 1 \end{vmatrix} = 17, \\
C_{12} &= - \begin{vmatrix} 2 & 4 \\ -1 & 1 \end{vmatrix} = -6, \\
C_{13} &= + \begin{vmatrix} 2 & 5 \\ -1 & -3 \end{vmatrix} = -1,\\[6pt]
C_{21} &= - \begin{vmatrix} 1 & 10 \\ -3 & 1 \end{vmatrix} = - (1\cdot1 - 10\cdot(-3)) = -(1+30) = -31, \\
C_{22} &= + \begin{vmatrix} 3 & 10 \\ -1 & 1 \end{vmatrix} = 3\cdot1 - 10\cdot(-1) = 3+10 = 13,\\
C_{23} &= - \begin{vmatrix} 3 & 1 \\ -1 & -3 \end{vmatrix} = - (3\cdot(-3) - 1\cdot(-1)) = -(-9+1) = 8,\\[6pt]
C_{31} &= + \begin{vmatrix} 1 & 10 \\ 5 & 4 \end{vmatrix} = 1\cdot4 - 10\cdot5 = 4-50 = -46, \\
C_{32} &= - \begin{vmatrix} 3 & 10 \\ 2 & 4 \end{vmatrix} = - (3\cdot4 - 10\cdot2) = -(12-20) = 8, \\
C_{33} &= + \begin{vmatrix} 3 & 1 \\ 2 & 5 \end{vmatrix} = 3\cdot5 - 1\cdot2 = 15-2 = 13.
\end{align*}

Матрица алгебраических дополнений:
\[
C = \begin{pmatrix}
17 & -6 & -1 \\
-31 & 13 & 8 \\
-46 & 8 & 13
\end{pmatrix}.
\]

\textbf{Шаг 3. Построение присоединённой матрицы.}

Присоединённая матрица — это транспонированная матрица алгебраических дополнений:
\[
\operatorname{adj}(A) = C^{\top} = \begin{pmatrix}
17 & -31 & -46 \\
-6 & 13 & 8 \\
-1 & 8 & 13
\end{pmatrix}.
\]

\textbf{Шаг 4. Вычисление обратной матрицы.}

\[
A^{-1} = \frac{1}{\det A} \cdot \operatorname{adj}(A) = \frac{1}{35} \begin{pmatrix}
17 & -31 & -46 \\
-6 & 13 & 8 \\
-1 & 8 & 13
\end{pmatrix}.
\]

\subsection*{Проверка} 

\textbf{Вычисляем \(A \cdot A^{-1}\) поэлементно} \\
\(b_{11} = \frac{1}{35} (3\cdot 17 + 1\cdot (-6) + 10\cdot (-1)) = \frac{1}{35}(51 - 6 - 10) = \frac{35}{35} = 1\)  \\
\(b_{21} = \frac{1}{35} (2\cdot 17 + 5\cdot (-6) + 4\cdot (-1)) = \frac{1}{35}(34 - 30 - 4) = \frac{0}{35} = 0\)  \\
\(b_{31} = \frac{1}{35} ((-1)\cdot 17 + (-3)\cdot (-6) + 1\cdot (-1)) = \frac{1}{35}(-17 + 18 - 1) = \frac{0}{35} = 0\)\\
\(b_{12} = \frac{1}{35} (3\cdot (-31) + 1\cdot 13 + 10\cdot 8) = \frac{1}{35}(-93 + 13 + 80) = \frac{0}{35} = 0\) \\ 
\(b_{22} = \frac{1}{35} (2\cdot (-31) + 5\cdot 13 + 4\cdot 8) = \frac{1}{35}(-62 + 65 + 32) = \frac{35}{35} = 1\) \\
\(b_{32} = \frac{1}{35} ((-1)\cdot (-31) + (-3)\cdot 13 + 1\cdot 8) = \frac{1}{35}(31 - 39 + 8) = \frac{0}{35} = 0\)\\
\(b_{13} = \frac{1}{35} (3\cdot (-46) + 1\cdot 8 + 10\cdot 13) = \frac{1}{35}(-138 + 8 + 130) = \frac{0}{35} = 0\) \\
\(b_{23} = \frac{1}{35} (2\cdot (-46) + 5\cdot 8 + 4\cdot 13) = \frac{1}{35}(-92 + 40 + 52) = \frac{0}{35} = 0\) \\
\(b_{33} = \frac{1}{35} ((-1)\cdot (-46) + (-3)\cdot 8 + 1\cdot 13) = \frac{1}{35}(46 - 24 + 13) = \frac{35}{35} = 1\)\\

Итог умножения:
\[
A \cdot A^{-1} = \begin{pmatrix}
1 & 0 & 0 \\
0 & 1 & 0 \\
0 & 0 & 1
\end{pmatrix} = E.
\]

Проверка успешна.


\subsection*{Ответ:}
\[
A^{-1} = \frac{1}{35} \begin{pmatrix}
17 & -31 & -46 \\
-6 & 13 & 8 \\
-1 & 8 & 13
\end{pmatrix}.
\]

\end{document}
