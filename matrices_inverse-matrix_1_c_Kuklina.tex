\documentclass{article}[russian, english] 

\usepackage{amsmath, amsthm, amssymb, amsfonts}
\usepackage{thmtools}
\usepackage{graphicx}
\usepackage{setspace}
\usepackage{geometry}
\usepackage{float}
\usepackage{hyperref}
\usepackage[utf8]{inputenc}
\usepackage[english]{babel}
\usepackage{framed}
\usepackage[dvipsnames]{xcolor}
\usepackage{tcolorbox}
\usepackage[T2A]{fontenc}    
\usepackage{tikz}
\usepackage[utf8]{inputenc}
\usepackage{wrapfig}
\usepackage{subcaption}

\newcommand{\RomanNumeralCaps}[1]
    {\MakeUppercase{\romannumeral #1}}



\newcommand{\HRule}[1]{\rule{\linewidth}{#1}}

\declaretheoremstyle[name=Theorem,]{thmsty}
\declaretheorem[style=thmsty,numberwithin=section]{theorem}
\tcolorboxenvironment{theorem}{colback=LightGray}

\declaretheoremstyle[name=Proposition,]{prosty}
\declaretheorem[style=prosty,numberlike=theorem]{proposition}
\tcolorboxenvironment{proposition}{colback=LightOrange}

\declaretheoremstyle[name=Principle,]{prcpsty}
\declaretheorem[style=prcpsty,numberlike=theorem]{principle}
\tcolorboxenvironment{principle}{colback=LightGreen}

\setstretch{1.2}
\geometry{
    textheight=9in,
    textwidth=5.5in,
    top=1in,
    headheight=12pt,
    headsep=25pt,
    footskip=30pt
}
\usepackage[skip=15pt plus1pt, indent=20pt]{parskip}
\begin{document}

\section*{Задача}

Дана матрица
\[
A = 
\begin{pmatrix}
1 & 0 & 1 \\
0 & 1 & 1 \\
1 & 1 & 1
\end{pmatrix}
\]
Найти обратную матрицу


\subsection*{Решение}

Способ 1:

Формула:  
\[
A^{-1} = \frac{1}{\det A} \cdot \operatorname{adj} A,
\]

\[
\det A = 1 \cdot (1 \cdot 1 - 1 \cdot 1) - 0 \cdot (0 \cdot 1 - 1 \cdot 1) + 1 \cdot (0 \cdot 1 - 1 \cdot 1)
\]
\[
= 1 \cdot (1 - 1) - 0 + 1 \cdot (0 - 1)
\]
\[
= 0 + 0 - 1 = -1
\]
\[
\det A = -1 \neq 0 \quad\Rightarrow\quad \text{обратная существует}.
\]

$C_{11}$: \(M_{11} = 1\cdot 1 - 1\cdot 1 = 1-1=0\), \(C_{11} = 0\).

$C_{12}$: \(M_{12} = 0\cdot 1 - 1\cdot 1 = -1\), \(C_{12} = (-1)^{1+2}(-1) = (-1)(-1) = 1\).

$C_{13}$: \(M_{13} = 0\cdot 1 - 1\cdot 1 = -1\), \(C_{13} = (-1)^{1+3}(-1) = (1)(-1) = -1\).

$C_{21}$: \(M_{21} = 0\cdot 1 - 1\cdot 1 = -1\), \(C_{21} = (-1)^{2+1}(-1) = (-1)(-1) = 1\).

$C_{22}$: \(M_{22} = 1\cdot 1 - 1\cdot 1 = 0\), \(C_{22} = 0\).

$C_{23}$: \(M_{23} = 1\cdot 1 - 0\cdot 1 = 1\), \(C_{23} = (-1)^{2+3} \cdot 1 = (-1) \cdot 1 = -1\).

$C_{31}$: \(M_{31} = 0\cdot 1 - 1\cdot 1 = -1\), \(C_{31} = (-1)^{3+1}(-1) = 1\cdot (-1) = -1\).

$C_{32}$: \(M_{32} = 1\cdot 1 - 0\cdot 1 = 1\), \(C_{32} = (-1)^{3+2} \cdot 1 = (-1)\cdot 1 = -1\).

$C_{33}$: \(M_{33} = 1\cdot 1 - 0\cdot 0 = 1\), \(C_{33} = (-1)^{3+3} \cdot 1 = 1\).

Матрица алгебраических дополнений:
\[
C = \begin{pmatrix}
0 & 1 & -1 \\
1 & 0 & -1 \\
-1 & -1 & 1
\end{pmatrix}
\]

Присоединённая:
\[
adj(A) = C^T = \begin{pmatrix}
0 & 1 & -1 \\
1 & 0 & -1 \\
-1 & -1 & 1
\end{pmatrix}^T
= \begin{pmatrix}
0 & 1 & -1 \\
1 & 0 & -1 \\
-1 & -1 & 1
\end{pmatrix}
\]


Обратная матрица
\[
A^{-1} = \frac{1}{-1} \cdot \begin{pmatrix}
0 & 1 & -1 \\
1 & 0 & -1 \\
-1 & -1 & 1
\end{pmatrix}
= \begin{pmatrix}
0 & -1 & 1 \\
-1 & 0 & 1 \\
1 & 1 & -1
\end{pmatrix}
\]
Способ 2:

1) Пишем расширенную матрицу \([A | E]\):
\[
\left(
\begin{array}{ccc|ccc}
1 & 0 & 1 & 1 & 0 & 0 \\
0 & 1 & 1 & 0 & 1 & 0 \\
1 & 1 & 1 & 0 & 0 & 1
\end{array}
\right)
\]
2) Из третьей строки вычтем первую: \(R_3\rightarrow R_3 - R_1\):
\[
\left(
\begin{array}{ccc|ccc}
1 & 0 & 1 & 1 & 0 & 0 \\
0 & 1 & 1 & 0 & 1 & 0 \\
0 & 1 & 0 & -1 & 0 & 1
\end{array}
\right)
\]
2) Из третьей строки вычтем вторую: \(R_3 \rightarrow R_3 - R_2\):
\[
\left(
\begin{array}{ccc|ccc}
1 & 0 & 1 & 1 & 0 & 0 \\
0 & 1 & 1 & 0 & 1 & 0 \\
0 & 0 & -1 & -1 & -1 & 1
\end{array}
\right)
\]

3) Умножим третью строку на \(-1\):
\[
\left(
\begin{array}{ccc|ccc}
1 & 0 & 1 & 1 & 0 & 0 \\
0 & 1 & 1 & 0 & 1 & 0 \\
0 & 0 & 1 & 1 & 1 & -1
\end{array}
\right)
\]

4) \(R_1 \rightarrow R_1 - R_3\) (чтоб убрать 1 в (1,3)):
\[
\left(
\begin{array}{ccc|ccc}
1 & 0 & 0 & 0 & -1 & 1 \\
0 & 1 & 1 & 0 & 1 & 0 \\
0 & 0 & 1 & 1 & 1 & -1
\end{array}
\right)
\]

5) \(R_2 \rightarrow R_2 - R_3\):
\[
\left(
\begin{array}{ccc|ccc}
1 & 0 & 0 & 0 & -1 & 1 \\
0 & 1 & 0 & -1 & 0 & 1 \\
0 & 0 & 1 & 1 & 1 & -1
\end{array}
\right)
\]

Получили \([E | A^{-1}]\), где
\[
A^{-1} = \begin{pmatrix}
0 & -1 & 1 \\
-1 & 0 & 1 \\
1 & 1 & -1
\end{pmatrix}
\]

Ответ:
\[
\boxed{
A^{-1} = \begin{pmatrix}
0 & -1 & 1 \\
-1 & 0 & 1 \\
1 & 1 & -1
\end{pmatrix}
}
\]
\end{document}
