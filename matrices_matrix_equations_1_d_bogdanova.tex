\documentclass{article}
\usepackage{amsmath, amsthm, amssymb, amsfonts}
\usepackage{thmtools}
\usepackage{graphicx}
\usepackage{setspace}
\usepackage{geometry}
\usepackage{float}
\usepackage{hyperref}
\usepackage[utf8]{inputenc}
\usepackage[english]{babel}
\usepackage{framed}
\usepackage[dvipsnames]{xcolor}
\usepackage{tcolorbox}
\usepackage[T2A]{fontenc}    
\usepackage{tikz}
\usepackage[utf8]{inputenc}
\usepackage{wrapfig}
\usepackage{subcaption}
\begin{document}

Дано матричное уравнение:
\[
X \times A = B
\]
где
\[
A = \begin{pmatrix} 2 & 1 \\ -4 & -3 \end{pmatrix}, \quad 
B = \begin{pmatrix} 2 & 2 \\ 6 & 4 \end{pmatrix}.
\]

\section*{Решение}

\subsection*{1. Находим определитель матрицы \(A\)}
\[
\det A = (2)(-3) - (1)(-4) = -6 + 4 = -2.
\]

\subsection*{2. Находим обратную матрицу \(A^{-1}\)}
Для матрицы \(A = \begin{pmatrix} a & b \\ c & d \end{pmatrix}\) обратная матрица вычисляется по формуле:
\[
A^{-1} = \frac{1}{\det A} \begin{pmatrix} d & -b \\ -c & a \end{pmatrix}.
\]
Подставляем значения:
\[
A^{-1} = \frac{1}{-2} \begin{pmatrix} -3 & -1 \\ 4 & 2 \end{pmatrix} = \begin{pmatrix} \frac{3}{2} & \frac{1}{2} \\ -2 & -1 \end{pmatrix}.
\]

\subsection*{3. Находим матрицу \(X\)}
Умножаем обе части исходного уравнения справа на \(A^{-1}\):
\[
X \times A \times A^{-1} = B \times A^{-1} \quad \Rightarrow \quad X = B \times A^{-1}.
\]
Выполняем умножение:
\[
X = \begin{pmatrix} 2 & 2 \\ 6 & 4 \end{pmatrix} \begin{pmatrix} \frac{3}{2} & \frac{1}{2} \\ -2 & -1 \end{pmatrix} = 
\begin{pmatrix} 
2 \cdot \frac{3}{2} + 2 \cdot (-2) & 2 \cdot \frac{1}{2} + 2 \cdot (-1) \\ 
6 \cdot \frac{3}{2} + 4 \cdot (-2) & 6 \cdot \frac{1}{2} + 4 \cdot (-1) 
\end{pmatrix} = 
\begin{pmatrix} 
3 - 4 & 1 - 2 \\ 
9 - 8 & 3 - 4 
\end{pmatrix} = 
\begin{pmatrix} 
-1 & -1 \\ 
1 & -1 
\end{pmatrix}.
\]

\subsection*{4. Проверка}
Умножим найденную матрицу \(X\) на \(A\):
\[
X \times A = \begin{pmatrix} -1 & -1 \\ 1 & -1 \end{pmatrix} \begin{pmatrix} 2 & 1 \\ -4 & -3 \end{pmatrix} = 
\begin{pmatrix} 
(-1)\cdot 2 + (-1)\cdot (-4) & (-1)\cdot 1 + (-1)\cdot (-3) \\ 
1\cdot 2 + (-1)\cdot (-4) & 1\cdot 1 + (-1)\cdot (-3) 
\end{pmatrix} = 
\begin{pmatrix} 
-2 + 4 & -1 + 3 \\ 
2 + 4 & 1 + 3 
\end{pmatrix} = 
\begin{pmatrix} 
2 & 2 \\ 
6 & 4 
\end{pmatrix} = B.
\]
Уравнение выполнено, решение верно.

\subsection*{Ответ:}
\[
X = \begin{pmatrix} -1 & -1 \\ 1 & -1 \end{pmatrix}.
\]

\end{document}
